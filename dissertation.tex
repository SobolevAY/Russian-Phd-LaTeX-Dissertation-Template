%&preformat-disser
\RequirePackage[l2tabu,orthodox]{nag} % Раскомментировав, можно в логе получать рекомендации относительно правильного использования пакетов и предупреждения об устаревших и нерекомендуемых пакетах
% Формат А4, 14pt (ГОСТ Р 7.0.11-2011, 5.3.6)
\documentclass[a4paper,14pt,oneside,openany]{memoir}

\input{common/setup}            % общие настройки шаблона
\input{common/packages}         % Пакеты общие для диссертации и автореферата
\synopsisfalse                      % Этот документ --- не автореферат
\input{Dissertation/dispackages}    % Пакеты для диссертации
\input{Dissertation/userpackages}   % Пакеты для специфических пользовательских задач

\input{Dissertation/setup}      % Упрощённые настройки шаблона

\input{common/newnames}         % Новые переменные, для всего проекта

%%% Основные сведения %%%
\newcommand{\thesisAuthorLastName}{Сердюк}
\newcommand{\thesisAuthorOtherNames}{Константин Сергеевич}
\newcommand{\thesisAuthorInitials}{К.\,С.}
\newcommand{\thesisAuthor}             % Диссертация, ФИО автора
{%
    \texorpdfstring{% \texorpdfstring takes two arguments and uses the first for (La)TeX and the second for pdf
        \thesisAuthorLastName~\thesisAuthorOtherNames% так будет отображаться на титульном листе или в тексте, где будет использоваться переменная
    }{%
        \thesisAuthorLastName, \thesisAuthorOtherNames% эта запись для свойств pdf-файла. В таком виде, если pdf будет обработан программами для сбора библиографических сведений, будет правильно представлена фамилия.
    }
}
\newcommand{\thesisAuthorShort}        % Диссертация, ФИО автора инициалами
{\thesisAuthorInitials~\thesisAuthorLastName}
%\newcommand{\thesisUdk}                % Диссертация, УДК
%{\todo{xxx.xxx}}
\newcommand{\thesisTitle}              % Диссертация, название
{Программное обеспечение быстрой обработки \\ данных электрического и электромагнитного каротажа \\ на основе многопараметричных моделей \\ околоскважинного пространства}
\newcommand{\thesisSpecialtyNumber}    % Диссертация, специальность, номер
{25.00.10}
\newcommand{\thesisSpecialtyTitle}     % Диссертация, специальность, название
{геофизика, геофизические методы поисков полезных ископаемых}
\newcommand{\thesisDegree}             % Диссертация, ученая степень
{кандидата технических наук}
\newcommand{\thesisDegreeShort}        % Диссертация, ученая степень, краткая запись
{канд. техн. наук}
\newcommand{\thesisCity}               % Диссертация, город написания диссертации
{Новосибирск}
\newcommand{\thesisYear}               % Диссертация, год написания диссертации
{2018}
\newcommand{\thesisOrganization}       % Диссертация, организация
{Федеральное государственное бюджетное учреждение науки \\ Институт нефтегазовой геологии и геофизики им. А.А.Трофимука \\ Сибирского отделения Российской академии наук (ИНГГ СО РАН)}
\newcommand{\thesisOrganizationShort}  % Диссертация, краткое название организации для доклада
{ИНГГ СО РАН}

\newcommand{\thesisInOrganization}     % Диссертация, организация в предложном падеже: Работа выполнена в ...
{Федеральном государственном бюджетном учреждении науки Институте нефтегазовой геологии и геофизики им. А.А. Трофимука Сибирского отделения Российской академии наук (ИНГГ СО РАН)}

\newcommand{\supervisorFio}            % Научный руководитель, ФИО
{Соболев Андрей Юрьевич}
\newcommand{\supervisorRegalia}        % Научный руководитель, регалии
{канд. техн. наук}
\newcommand{\supervisorFioShort}       % Научный руководитель, ФИО
{А.\,Ю.~Соболев}
\newcommand{\supervisorRegaliaShort}   % Научный руководитель, регалии
{к.\,т.\,н.}


\newcommand{\opponentOneFio}           % Оппонент 1, ФИО
{\todo{Фамилия Имя Отчество}}
\newcommand{\opponentOneRegalia}       % Оппонент 1, регалии
{\todo{доктор физико-математических наук, профессор}}
\newcommand{\opponentOneJobPlace}      % Оппонент 1, место работы
{\todo{Не очень длинное название для места работы}}
\newcommand{\opponentOneJobPost}       % Оппонент 1, должность
{\todo{старший научный сотрудник}}

\newcommand{\opponentTwoFio}           % Оппонент 2, ФИО
{\todo{Фамилия Имя Отчество}}
\newcommand{\opponentTwoRegalia}       % Оппонент 2, регалии
{\todo{кандидат физико-математических наук}}
\newcommand{\opponentTwoJobPlace}      % Оппонент 2, место работы
{\todo{Основное место работы c длинным длинным длинным длинным названием}}
\newcommand{\opponentTwoJobPost}       % Оппонент 2, должность
{\todo{старший научный сотрудник}}

\newcommand{\leadingOrganizationTitle} % Ведущая организация, дополнительные строки
{Федеральное государственное бюджетное учреждение науки Институт нефтегазовой геологии и геофизики им. А.А.Трофимука Сибирского отделения Российской академии наук (ИНГГ СО РАН)}

\newcommand{\defenseDate}              % Защита, дата
{\todo{DD mmmmmmmm YYYY~г.~в~XX часов}}
\newcommand{\defenseCouncilNumber}     % Защита, номер диссертационного совета
{Д\,003.068.03}
\newcommand{\defenseCouncilTitle}      % Защита, учреждение диссертационного совета
{Федеральном государственном бюджетном учреждении науки Институте нефтегазовой геологии и геофизики им. А.А. Трофимука Сибирского отделения Российской академии наук, в конференц-зале}
\newcommand{\defenseCouncilAddress}    % Защита, адрес учреждение диссертационного совета
{630090, пр-т Ак. Коптюга, 3, г. Новосибирск}
\newcommand{\defenseCouncilPhone}      % Телефон для справок
{+7 (383) 333-25-13}

\newcommand{\defenseSecretaryFio}      % Секретарь диссертационного совета, ФИО
{Неведрова Нина Николаевна}
\newcommand{\defenseSecretaryRegalia}  % Секретарь диссертационного совета, регалии
{\todo{д. г.-м. н.}}            % Для сокращений есть ГОСТы, например: ГОСТ Р 7.0.12-2011 + http://base.garant.ru/179724/#block_30000

\newcommand{\synopsisLibrary}          % Автореферат, название библиотеки
{ИНГГ СО РАН}
\newcommand{\synopsisDate}             % Автореферат, дата рассылки
{\todo{DD mmmmmmmm YYYY года}}

% To avoid conflict with beamer class use \providecommand
\providecommand{\keywords}%            % Ключевые слова для метаданных PDF диссертации и автореферата
{}
             % Основные сведения
\input{common/fonts}            % Определение шрифтов (частичное)
\input{common/styles}           % Стили общие для диссертации и автореферата
\input{Dissertation/disstyles}  % Стили для диссертации
\input{Dissertation/userstyles} % Стили для специфических пользовательских задач

%%% Библиография. Выбор движка для реализации %%%
\ifnumequal{\value{bibliosel}}{0}{%
    \input{biblio/predefined}   % Встроенная реализация с загрузкой файла через движок bibtex8
}{
    \input{biblio/biblatex}     % Реализация пакетом biblatex через движок biber
}

%%% Управление компиляцией отдельных частей диссертации %%%
% Необходимо сначала иметь полностью скомпилированный документ, чтобы все
% промежуточные файлы были в наличии
% Затем, для вывода отдельных частей можно воспользоваться командой \includeonly
% Ниже примеры использования команды:
%
%\includeonly{Dissertation/part2}
%\includeonly{Dissertation/contents,Dissertation/appendix,Dissertation/conclusion}
%
% Если все команды закомментированы, то документ будет выведен в PDF файл полностью

\begin{document}

\input{common/renames}                 % Переопределение именований

%%% Структура диссертации (ГОСТ Р 7.0.11-2011, 4)
\include{Dissertation/title}           % Титульный лист
\include{Dissertation/contents}        % Оглавление
\chapter*{Введение}                         % Заголовок
\addcontentsline{toc}{chapter}{Введение}    % Добавляем его в оглавление

\newcommand{\actuality}{}
\newcommand{\progress}{}
\newcommand{\aim}{{\textbf\aimTXT}}
\newcommand{\tasks}{\textbf{\tasksTXT}}
\newcommand{\novelty}{\textbf{\noveltyTXT}}
\newcommand{\influence}{\textbf{\influenceTXT}}
\newcommand{\methods}{\textbf{\methodsTXT}}
\newcommand{\defpositions}{\textbf{\defpositionsTXT}}
\newcommand{\reliability}{\textbf{\reliabilityTXT}}
\newcommand{\probation}{\textbf{\probationTXT}}
\newcommand{\contribution}{\textbf{\contributionTXT}}
\newcommand{\publications}{\textbf{\publicationsTXT}}


{\actuality} Обзор, введение в тему, обозначение места данной работы в
мировых исследованиях и~т.\:п., можно использовать ссылки на~другие
работы\ifnumequal{\value{bibliosel}}{1}{~\autocite{Gosele1999161}}{}
(если их~нет, то~в~автореферате
автоматически пропадёт раздел <<Список литературы>>). Внимание! Ссылки
на~другие работы в разделе общей характеристики работы можно
использовать только при использовании \verb!biblatex! (из-за технических
ограничений \verb!bibtex8!. Это связано с тем, что одна
и~та~же~характеристика используются и~в~тексте диссертации, и в
автореферате. В~последнем, согласно ГОСТ, должен присутствовать список
работ автора по~теме диссертации, а~\verb!bibtex8! не~умеет выводить в одном
файле два списка литературы).
При использовании \verb!biblatex! возможно использование исключительно
в~автореферате подстрочных ссылок
для других работ командой \verb!\autocite!, а~также цитирование
собственных работ командой \verb!\cite!. Для этого в~файле
\verb!Synopsis/setup.tex! необходимо присвоить положительное значение
счётчику \verb!\setcounter{usefootcite}{1}!.

Для генерации содержимого титульного листа автореферата, диссертации
и~презентации используются данные из файла \verb!common/data.tex!. Если,
например, вы меняете название диссертации, то оно автоматически
появится в~итоговых файлах после очередного запуска \LaTeX. Согласно
ГОСТ 7.0.11-2011 <<5.1.1 Титульный лист является первой страницей
диссертации, служит источником информации, необходимой для обработки и
поиска документа>>. Наличие логотипа организации на~титульном листе
упрощает обработку и~поиск, для этого разметите логотип вашей
организации в папке images в~формате PDF (лучше найти его в векторном
варианте, чтобы он хорошо смотрелся при печати) под именем
\verb!logo.pdf!. Настроить размер изображения с логотипом можно
в~соответствующих местах файлов \verb!title.tex!  отдельно для
диссертации и автореферата. Если вам логотип не~нужен, то просто
удалите файл с~логотипом.

\ifsynopsis
Этот абзац появляется только в~автореферате.
Для формирования блоков, которые будут обрабатываться только в~автореферате,
заведена проверка условия \verb!\!\verb!ifsynopsis!.
Значение условия задаётся в~основном файле документа (\verb!synopsis.tex! для
автореферата).
\else
Этот абзац появляется только в~диссертации.
Через проверку условия \verb!\!\verb!ifsynopsis!, задаваемого в~основном файле
документа (\verb!dissertation.tex! для диссертации), можно сделать новую
команду, обеспечивающую появление цитаты в~диссертации, но~не~в~автореферате.
\fi

% {\progress} 
% Этот раздел должен быть отдельным структурным элементом по
% ГОСТ, но он, как правило, включается в описание актуальности
% темы. Нужен он отдельным структурынм элемементом или нет ---
% смотрите другие диссертации вашего совета, скорее всего не нужен.

{\aim} данной работы является \ldots

Для~достижения поставленной цели необходимо было решить следующие {\tasks}:
\begin{enumerate}
  \item Исследовать, разработать, вычислить и~т.\:д. и~т.\:п.
  \item Исследовать, разработать, вычислить и~т.\:д. и~т.\:п.
  \item Исследовать, разработать, вычислить и~т.\:д. и~т.\:п.
  \item Исследовать, разработать, вычислить и~т.\:д. и~т.\:п.
\end{enumerate}


{\novelty}
\begin{enumerate}
  \item Впервые \ldots
  \item Впервые \ldots
  \item Было выполнено оригинальное исследование \ldots
\end{enumerate}

{\influence} \ldots

{\methods} \ldots

{\defpositions}
\begin{enumerate}
  \item Первое положение
  \item Второе положение
  \item Третье положение
  \item Четвертое положение
\end{enumerate}
В папке Documents можно ознакомиться в решением совета из Томского ГУ
в~файле \verb+Def_positions.pdf+, где обоснованно даются рекомендации
по~формулировкам защищаемых положений. 

{\reliability} полученных результатов обеспечивается \ldots \ Результаты находятся в соответствии с результатами, полученными другими авторами.


{\probation}
Основные результаты работы докладывались~на:
перечисление основных конференций, симпозиумов и~т.\:п.

{\contribution} Автор принимал активное участие \ldots

%\publications\ Основные результаты по теме диссертации изложены в ХХ печатных изданиях~\cite{Sokolov,Gaidaenko,Lermontov,Management},
%Х из которых изданы в журналах, рекомендованных ВАК~\cite{Sokolov,Gaidaenko}, 
%ХХ --- в тезисах докладов~\cite{Lermontov,Management}.

\ifnumequal{\value{bibliosel}}{0}{% Встроенная реализация с загрузкой файла через движок bibtex8
    \publications\ Основные результаты по теме диссертации изложены в XX печатных изданиях, 
    X из которых изданы в журналах, рекомендованных ВАК, 
    X "--- в тезисах докладов.%
}{% Реализация пакетом biblatex через движок biber
%Сделана отдельная секция, чтобы не отображались в списке цитированных материалов
    \begin{refsection}[vak,papers,conf]% Подсчет и нумерация авторских работ. Засчитываются только те, которые были прописаны внутри \nocite{}.
        %Чтобы сменить порядок разделов в сгрупированном списке литературы необходимо перетасовать следующие три строчки, а также команды в разделе \newcommand*{\insertbiblioauthorgrouped} в файле biblio/biblatex.tex
        \printbibliography[heading=countauthorvak, env=countauthorvak, keyword=biblioauthorvak, section=1]%
        \printbibliography[heading=countauthorconf, env=countauthorconf, keyword=biblioauthorconf, section=1]%
        \printbibliography[heading=countauthornotvak, env=countauthornotvak, keyword=biblioauthornotvak, section=1]%
        \printbibliography[heading=countauthor, env=countauthor, keyword=biblioauthor, section=1]%
        \nocite{%Порядок перечисления в этом блоке определяет порядок вывода в списке публикаций автора
                Serdyuk2011,Serdyuk2014,Serdyuk2017,%
                Baranova2014,Koshelev2012,Serdyuk2013a,Teit2011,Uramaev2015,%
                bib1,bib2,%
        }%
        \publications\ Основные результаты по теме диссертации изложены в~\arabic{citeauthor}~печатных изданиях, 
        \arabic{citeauthorvak} из которых изданы в журналах, рекомендованных ВАК, 
        \arabic{citeauthorconf} "--- в~тезисах докладов.
    \end{refsection}
    \begin{refsection}[vak,papers,conf]%Блок, позволяющий отобрать из всех работ автора наиболее значимые, и только их вывести в автореферате, но считать в блоке выше общее число работ
        \printbibliography[heading=countauthorvak, env=countauthorvak, keyword=biblioauthorvak, section=2]%
        \printbibliography[heading=countauthornotvak, env=countauthornotvak, keyword=biblioauthornotvak, section=2]%
        \printbibliography[heading=countauthorconf, env=countauthorconf, keyword=biblioauthorconf, section=2]%
        \printbibliography[heading=countauthor, env=countauthor, keyword=biblioauthor, section=2]%
        \nocite{vakbib2}%vak
        \nocite{bib1}%notvak
        \nocite{confbib1}%conf
    \end{refsection}
}
При использовании пакета \verb!biblatex! для автоматического подсчёта
количества публикаций автора по теме диссертации, необходимо
их~здесь перечислить с использованием команды \verb!\nocite!.
 % Характеристика работы по структуре во введении и в автореферате не отличается (ГОСТ Р 7.0.11, пункты 5.3.1 и 9.2.1), потому её загружаем из одного и того же внешнего файла, предварительно задав форму выделения некоторым параметрам

\textbf{Объем и структура работы.} Диссертация состоит из~введения, четырёх глав,
заключения и~двух приложений.
%% на случай ошибок оставляю исходный кусок на месте, закомментированным
%Полный объём диссертации составляет  \ref*{TotPages}~страницу
%с~\totalfigures{}~рисунками и~\totaltables{}~таблицами. Список литературы
%содержит \total{citenum}~наименований.
%
Полный объём диссертации составляет
\formbytotal{TotPages}{страниц}{у}{ы}{}, включая
\formbytotal{totalcount@figure}{рисун}{ок}{ка}{ков} и
\formbytotal{totalcount@table}{таблиц}{у}{ы}{}.   Список литературы содержит
\formbytotal{citenum}{наименован}{ие}{ия}{ий}.
    % Введение
\chapter{Обзор существующих методик решения практических и теоретических решений прямых и обратных задач скважинной геофизики} \label{chapt1}

Тут я процитирую работы чтобы появились в списке литературы~\cite{Serdyuk2011,Serdyuk2014,Serdyuk2017,Baranova2014,Koshelev2012,Serdyuk2013a,Teit2011,Uramaev2015}.

Для компенсации избыточного давления стенок скважины в процессе бурения и
проведения каротажа используют буровой раствор. Он также применяется для
промывки скважины от бурового шлама. Буровые растворы могут быть сделаны на
водной, углеводной и аэрированной основах [5]. Выбор основы зависит от типа
бурящихся пород. Фильтрат бурового раствора проникает в породу, образуя
зоны с различными УЭС и ОДП. Количество таких зон зависит от фильтрационных
свойств среды.

Структура пластов может быть достаточно неоднородной, и для ее описания
потребовалось бы большое количество параметров. Поэтому вместо среды
используют приближённую модель, которая описывает основные свойства с точки
зрения рассматриваемого метода исследования.

При интерпретации данных электрических и электромагнитных методов в каждой
точке измерений среда моделируется набором радиальных слоев с
концентрическими цилиндрическими границами и с постоянными значениями
параметров внутри каждого слоя. Для метода ВИКИЗ и БК искомые параметры
"--- УЭС и ширины слоёв. Эти параметры являются ключевыми, но кроме того,
на показания зондов ВИКИЗ в некоторых случаях влияет ОДП.

Наиболее приближенной к реальной описывают среду трех- и четырехслойные
модели.  Трехслойная модель (рис. 1) состоит из скважины, зоны
проникновения (ЗП) и нетронутой буровым раствором части пласта.
Четырехслойная модель (рис. 2) описывает ситуацию, когда проникший в породу
фильтрат бурового раствора образует зону проникновения и окаймляющую зону
(ОЗ).

Для задач скважинной геомеханики основная задача состоит в определении
механических свойств породы и нахождения устойчивого состояния стенок
скважины. Как правило модели предполагают наличие частичной симметричности
относительно оси скважины, но не учитывают наличие радиальных слоев помимо
самой скважины. Для таких задач обычно встает вопрос поиска корреляционных
зависимостей между данными исследований образцов КЕРНа и данными ГИС.

В процессе интерпретации каротажа происходит решение прямых и обратных
задач. Прямая задача заключается в определении показаний прибора,
находящегося в заданной модели среды. Обратная же задача, наоборот, состоит
в определении среды, в которой величины измерений прибора совпадают с
заданными.

\section{Походы к решению задач: палетки, оптимизационные методы, нейросетевые} \label{sect1_1}

\subsection{Аналитические решатели} \label{subsect1_1_1}
Для электрических и электромагнитных методов существуют алгоритмы точного
решения прямых задач. Они основаны на законах Максвелла. Такие алгоритмы
зачастую работают медленно.

Для ВИКИЗ, одним из точных алгоритмов является ALVIK [7], разработанный и
реализованный в ИГиГ СО АН СССР. Особенность данного алгоритма состоит в
том, что он позволяет решать прямые задачи для произвольных конфигураций
прибора типа ВИКИЗ. Скорость решения одной одномерной трёхслойной задачи
составляет 90 мс, а четырёхслойной "---  120 мс.

Представителем реализованных алгоритмов точного решения для бокового
каротажа является реализации Миронцова, сотрудника ИНГГ СО РАН. Этот
алгоритм был разработан специально для моделирования прибора,
разработанного в ЗАО НПП ГА «Луч». Он предназначен для решения трёхслойных
одномерных задач. Скорость решения такой прямой задачи на порядок ниже, чем
у ВИКИЗ "--- порядка двух секунд.

\subsection{Классический палеточный подход} \label{subsect1_1_2}
Решение обратной задачи на основе палеток осуществлялось еще до широкого
распространения ЭВМ. Сперва насчитывались прямые задачи и по результатам
строились графики зависимости сигнала прибора от сопротивления однородной
среды. Такие  графики наносились на кальку. Решение обратной задачи
сводилось к наложению кальки к каротажным кривым. Задача считалась
решенной, если кривые совпадали с некоторой точностью. Несмотря на
архаичность данного метода, он до сих пор используется в некоторых
нефтедобывающих компаниях.

Более современные реализации палеток – совокупность заранее насчитанных
прямых задач и их входных данных. Решение обратной задачи на таких палетках
сводится к выборке входных данных прямой задачи по результатам её решения.
Палетками также может служить нейронная сеть, обученная на множестве
решений прямых задач [8].

Существующие реализации решения обратных задач на основе палеток работают с
двух- или трёхслойными одномерными моделями. Причем трёхслойные модели
имеют фиксированные значения некоторых параметров. Это определяется
необходимостью иметь большое количество памяти и вычислительных ресурсов.
Так же можно отметить, что область покрытия существующими палетками
необходимых моделей недостаточна.

\subsection{Нейросети} \label{subsect1_1_3}

\subsection{Оптимизационные алгоритмы} \label{subsect1_1_4}

Другой комплекс методов решения обратной задачи "--- это итерационные
методы. Схема решения  представлена на рис. 5. На основе полученных в ходе
зондирования показаний прибора строится стартовая модель, которая подается
на вход прямой задачи для вычисления теоретических показаний и расчета
невязки. Если невязка меньше некоторого заданного значения, можно считать
решение удовлетворительным. Если для текущей модели теоретические и
измеренные показания дают большое расхождение, модель корректируется и
процесс повторяется. Т.е. решение обратной задачи сводится к многократному
решению прямой задачи. Поэтому критично, чтобы скорость решения прямой
задачи была как можно выше.


\section{Существующее ПО для интерпретации} \label{sect1_2}

Для интерпретации данных электромагнитного каротажа большинство российских
компаний используют лишь небольшое количество различных комплексных
программных систем. Среди таких систем Techlog [4] от компании
Schlumberger, СИАЛ-ГИС [5], ПРАЙМ [6], «ГеоПоиск» [7], Geo Office Solver
[8]. Другие продукты не рассматриваются ввиду малой распространенности на
территории Западной Сибири.

Рассмотрим функционал представленных программных систем с точки зрения
средств визуализации, хранения и обработки данных электрометрических
исследований в скважинах.

\subsection{ПРАЙМ} \label{subsect_Prime}
Программная система ПРАЙМ является автоматизированным инструментом в среде
Windows для сбора, визуализации, обработки и хранения геолого-геофизических
данных в нефтегазовой отрасли. Система имеет модульную структуру, которая
позволяет легко собирать рабочие места различной сложности и назначения для
удовлетворения потребностей конкретного пользователя. Такая структура
позволяет обеспечить масштабируемость от мелких компаний до крупных
корпораций.

Организация данных и ее функциональные возможности позволяют решать задачи
сбора, анализа и обработки данных ГИС на всех этапах жизни скважины:
открытого ствола, цементирования и обсаженного ствола.

ПРАЙМ является не только готовой системой для практического использования,
но и универсальной средой для разработки новых приложений. Открытость для
расширения и развития в стандарте DLL, COM и внутреннего языка позволяют
пользователям самостоятельно развивать возможности системы для адаптации к
конкретным условиям. Расширения системы ПРАЙМ представляют собой модули
(DLL), которые могут быть подключены к системе в любой момент без
привлечения разработчиков системы. Допускаются любые сочетания расширений.
Каждое расширение системы ПРАЙМ обычно позволяют решить отдельную задачу
обработки данных ГИС.

Модули системы Прайм условно разделены по блокам. Сервер данных и сервер
планшета обеспечивают разработчиков около 1000 универсальными готовыми
функциями для разработки приложений.

Оригинальная технология построения и управления локальной базой данных
обеспечивает необходимую универсальность и гибкость при импорте-экспорте и
построении технологий обработки данных на местах. Отличается от других
систем тем, что не навязывает собственную модель данных или внутренний
формат представления каротажных кривых.

Поддерживает как непрерывные, так и попластовые типы данных и технологии их
обработки, в том числе и для смешанных типов данных. Имеются специальные
средства для работы с данными керна. Пользователь может самостоятельно
создавать или пользоваться готовыми графами обработки данных.

\subsection{ГеоПоиск} \label{subsect_Poisk}
ГеоПоиск "--- это программный пакет в среде Windows, предназначенный для
обработки и интерпретации данных геофизического исследования скважин с
привлечением смежной информации на уровнях от отдельной скважины, куста
скважин до месторождения и группы месторождений.

Система имеет собственную общую структурированную геофизическую базу
данных. Все коммуникации между различными модулями системы осуществляются
через эту БД. Существует возможность потоковой обработки множества скважин.
Для этого в программный комплекс встроена следующая функциональность:
потоковая загрузка в базу с автоматическим созданием и заполнением скважин
данными; выполнение алгоритмов (формул) для группы выбранных скважин;
создание текстовых отчетов-таблиц любой сложности по данным группы скважин,
месторождению; «однокнопочное» формирование планшета скважины по шаблону
любой сложности. Для визуализации данных используются подход, основанный на
макетном (шаблонном) представлении.

\subsection{GeoOffice Solver} \label{subsect_Solver}
Программный комплекс GeoOffice Solver предназначено для сбора,
архивирования и интерпретации данных ГИС. Он включает в себя все
необходимые вычислительные алгоритмы для предобработки и интерпретации
данных БКЗ, визуализирует полученную геоэлектрическую модель. Для обработки
данных ВИКИЗ, система GeoOffice Solver интегрирована с комплексом МФС
ВИКИЗ.

\subsection{СИАЛ-ГИС} \label{subsect_Sial}
СИАЛ-ГИС "--- автоматизированная система обработки и интерпретации данных
каротажа скважин, ориентированная на промышленную эксплуатацию как в
геофизических предприятиях для оперативной интерпретации и выдачи
окончательного заключения по скважинам, так и в научно-исследовательских и
проектных институтах для интерпретации материалов ГИС при пересчете запасов
и построении постоянно действующих геологических и гидродинамических
моделей месторождений.

Система реализует в полном объеме все этапы традиционной интерпретации
данных каротажа. Алгоритмы системы являются результатом формализации
классических приемов и способов, пригодных для любого района. Кроме того,
система содержит решения специфических задач, актуальных в Западной Сибири,
с учетом специфики существующего ограниченного комплекса измерений в
эксплуатационных скважинах и геолого-геофизических особенностей района
работ.

Система имеет, в качестве основного, вариант полностью непрерывного
автоматизированного процесса интерпретации по скважине от загрузки
каротажных кривых на входе, до формирования заключения на выходе,
содержащего полную информацию о литологии пластов, о наличии коллекторов в
обрабатываемом разрезе, о характере и степени их насыщения,
фильтрационно-емкостных свойствах. В то же время система предоставляет
широкие возможности контроля и корректировки исходных данных, промежуточных
и окончательных результатов в интерактивно-графическом режиме, позволяющие
интерпретатору по желанию вмешиваться в процесс обработки, принимая в ней
творческое участие на наиболее ответственных этапах.

Обработка данных электрометрии по ВИКИЗ построена с использованием модуля
МФС ВИКИЗ, поставляемого ИНГГ СО РАН. Стыковка произведена по следующей
схеме: в модуль МФС из СИАЛ-ГИС подаются литологические границы. В самом
МФС ВИКИЗе производится автоматическое снятие отсчетов по алгоритмам,
принятым для ВИКИЗ, добавляются омические границы в интерактивном режиме и
производится определение Рп обычным для этой программы способом. По
завершении этой работы пользователь просто выходит из модуля ВИКИЗ и
попадает обратно в СИАЛ-ГИС. Вся информация, полученная в ВИКИЗ,
сохраняется в СИАЛ-ГИС и реэкспортируется в ВИКИЗ при желании пересмотреть
или уточнить Рп по ВИКИЗ.

\subsection{Techlog} \label{subsect_Techlog}
Techlog "--- крупный программный геофизический комплекс, разработанный
компанией Schlumberger. Он предназначен для обработки и интерпретации
каротажных данных. В спектр его сервисов входит импорт LAS-файлов, хранение
каротажных кривых, визуализация данных, набор математических алгоритмов
трансформаций и др. Данный продукт позволяет создавать собственные
алгоритмы и интегрировать сторонние модули, что позволяет расширить набор
предлагаемых возможностей. Механизм интеграции представлен в виде
исполняемых скриптов, написанных на языке Python. В Techlog встроен
собственный Python редактор с отладчиком. Программные модули встраиваются в
панель инструментов в виде меню, написанных на языке xml. Как было отмечено
ранее, отличительной особенностью Techlog является предоставление
возможности автоматической интерпретации данных. Пользователь может создать
так называемый Python Workflow Item (Python AWI), который объединит
необходимые скрипты в исполняемый непрерывно конвейер. На вход такого
процесса могут быть подано сразу множество однотипных данных, полученных в
результате исследований в различных скважинах.

           % Глава 1
\chapter{Палеточный подход} \label{chapt2}

В статье предлагается схема формирования и использования палеток для
различных задач одномерного моделирования, используя современные
вычислительные мощности. Это позволяет обойти упомянутые выше недостатки
существующих реализаций палеточного подхода. В данной работе
рассматриваются прямые и обратные задачи ВИКИЗ и БК на одномерных
многопараметричных моделях, что, впрочем, не ограничивает общности подхода.
Такие модели представлены набором вложенных цилиндрических колец (слоёв) с
бесконечной шириной внешнего слоя (рис.1). Каждый слой имеет однородное
пространственное распределение относительной диэлектрической проницаемости
(ОДП) и удельного электрического сопротивления (УЭС).

В общем виде схема формирования палеток состоит из пяти взаимосвязанных
процедур (этапов), образующих итеративный процесс (рис. 2).


\section{Выбор расчетной сетки} \label{sect2_1}

Целью предварительного расчета прямых задач является ускорение за счет
быстрого поиска. Наиболее быстро поиск осуществляется при регулярно
организованных данных. Скорость поиска по таким данным мало зависит от
количества данных, а сам поиск сводится к выборке по конкретному адресу.
Это возможно осуществить, если закодировать входные данные одним числом
"--- порядковым номером в конечномерной расчетной сети.

Одной из главных характеристик расчетной сети является величина шага.
Исходя из природы электромагнитных методов, целесообразно рассматривать
регулярную сеть с экспоненциальным шагом узлов.

Диапазон значений параметров среды желательно выбирать как можно шире.
Число узлов в заданной расчетной сети зависит от имеющих вычислительных
ресурсов и скорости вычисления точного решения прямой задачи. Следует
учитывать, что есть узлы, для которых желательно получать более точные
решения с высокой скоростью. Для этого расчетную сеть можно разделить  на
несколько пересекающихся по нужным параметрам.

\section{Проведение расчета на высокопроизводительной системе} \label{sect2_2}

Для случая, когда время вычисления задач на одной вычислительной машине для
заданной расчетной сети неприемлемо велико, разработан инструмент для
разбиения всего множества задач на подзадачи заданного размера. Это
необходимо для автоматизированного расчета на множестве вычислительных
узлов. Такими узлами могут выступать элементы грид-системы [10].
Преимущество такой системы в том, что она идеально подходит для расчета
задач на сетке. При этом на программу точного (аналитического) решения не
накладывается особых условий. Тем не менее, желательно, чтобы за один
запуск программы рассчитывалась не один узел сети, а некоторое её
подмножество. Это условие происходит из того, что время на передачу данных
должно быть меньше времени расчета на одной машине.

Преимуществом грид-систем перед кластерами в возможности запускать
имеющиеся бинарные программы для ОС Windows на рабочих станциях без
перекомпиляции кода.

В качестве грид-системы был выбран HTCondor [11]. Его преимущества
заключается в том, что он бесплатный, достаточно часто обновляется и
работает как на Unix подобных ОС, так и на ОС семейства Windows.

\section{Оптимизация сетки} \label{sect2_3}

Преимущества регулярной сети могут частично нивелироваться в случае, если
количество выходных данных слишком велико. В частности, количество
измеряемых данных высокочастотными индукционными приборами может достигать
двух десятков штук на каждый узел сети. Размер результирующих данных для
сети, покрывающей рассматриваемую область входных параметров с достаточной
точностью при аппроксимации по ним, составляет около четырех гигабайт. Это
означает, что время позиционирования по данным будет больше, чем получение
результата имеющимся решателем.

Исходя из вышеизложенного, был разработан оптимизирующий данные модуль, в
задачи которого входят уменьшение количества данных, при сохранении
локальной регулярности и возможности аппроксимации по этим данным с
требуемой точностью.

Суть лежащего в основе инструмента алгоритма заключается в удалении узлов
исходной сети по следующему правилу. Считается, что узел можно удалить,
если смоделированный в нём сигнал интерполируется сигналами соседних узлов
и разница между ними находится в пределах заданной погрешности. При этом
допускается погрешность равная погрешности измерений прибора. Интерполяция
проводится полиномом второй степени, построенным через три ближайшие точки.
В процессе оптимизации, узлы удаляются таким образом, что сеть оказывалась
разбитой на части, далее зоны, имеющие регулярную структуру. Такой подход
позволяет: во-первых, сокращать объём сети, сохраняя быстрый доступ к её
узлам благодаря регулярности зон, и, во-вторых, допускает уточнение
поведения сигналов на некоторой области моделей, путём пересчёта зоны, в
которую входит интересующая нас область, с более мелким шагом.

На рисунках ниже (рис. 1, рис. 2) представлена программная реализация
модуля оптимизации насчитанных на грид-системе данных. Данные находятся в
нескольких файлах, по числу подзадач, на которые разбита расчетная сеть. В
каждом файле результат расчета прямой задачи на некотором связанном
подмножестве моделей (т.е. на n-мерном параллелепипеде, где n-количество
параметров модели среды). Для работы с результатами расчетов, как с единым
пространством служит класс GridDataController. Класс GridZone хранит
информацию о пространстве моделей: геометрию и значения параметров модели
среды. Класс, ответственный за оптимизацию, "--- GirdOptimizeController.

Отношения рассмотренных трех сущностей [12] (GridDataController, GridZone,
GirdOptimizeController) составляют основу модуля по оптимизации насчитанных
данных (рис. 1).

Ниже представлена функциональная схема модуля по оптимизации насчитанных
данных (рис. 2).


\section{Организация в единое хранилище в зависимости от решаемой задачи} \label{sect2_4}

Для организации полученных результатов были разработаны форматы
представления для двух способов хранения: в бинарном файле и в базе данных.
Во втором случае была использована база данных SQLite в виду ряда важных
преимуществ, таких как:
\begin{itemize}
\item она свободна;
\item API доступен в виде отдельного С файла без внешних зависимостей;
\item поддержка хранения больших объёмов данных: до терабайта; 
\item относительно быстро происходит выборка данных из таблиц;
\item отсутствие сервера при работе с базой данных.
\end{itemize}

Плюсы использования базы данных в качестве хранилища – простота изменения и
добавления данных.  Хранение данных непосредственно в бинарном файле
уменьшает время выборки и объём, но теряется гибкость работы с содержимым.

Программный модуль работы с базой данных включает функции создания,
добавления и удаления регулярных подпространств. На рисунке (рис. 3)
приведена структура базы данных SQLite этого модуля, т.е. таблицы и связи
между ними. Таблицы проектировались таким образом, чтобы такие операции как
соединение, добавление и удаление подобластей, на которые разбиты данные,
выполнялись быстро.

Таблица Properties содержит информацию о свойствах палеток, таких как
точность представления данных, идентификатор алгоритма решателя, параметры
прибора.  Модель представляется одним числом (поле id в таблице Model).
Есть отдельная таблица (SubSpace) для хранения подпространств рассчитанных
данных. Т.к. подпространство – это многомерный параллелепипед, то в таблице
хранятся только координаты двух его вершин. Для нахождения моделей близких
к заданной в таблице SubSpace хранятся начальный (id\_b) и конечный (id\_e)
идентификаторы моделей. Каждый сигнал имеет свой уникальный номер, который
хранится в поле id таблицы Signal. Для увеличения скорости нахождения
сигналов, близких к заданному, существует дополнительная таблица
Signal\_marks.

Модуль организации данных палетки в бинарном файле создает три секции,
которые содержат:
\begin{itemize}
\item информацию о подпространствах (секция SSI);
\item информацию о связи моделей и подпространств (MI);
\item информацию о сигналах и их связях с моделями (SI).
\end{itemize}

Секция SSI состоит из совокупности векторов. Эти вектора содержат
уникальный идентификатор подпространства и информацию об этом
подпространстве. Эти вектора упорядочены по идентификатору в
лексикографическом порядке.

Идентификатор сигнала равен номеру его вхождения в SI. Секция MI состоит из
множества пар, которые содержат идентификатор подпространства из SSI и
идентификатор сигналов. Идентификатор модели соответствует порядку
расположения этих пар в MI. Секция SI в свою очередь состоит и пар "--- 
идентификатор модели и сигнал.

\section{Поиск решений и аппроксимации по данным палетки} \label{sect2_5}

Разработанный модуль поиска может решать две задачи.

\subsection{Прямая задача}

Первая задача заключается в нахождении ближайших данных, соответствующих
заданной модели, аппроксимации по ним. Такую задачу назовем прямой задачей.
При решении прямой задачи происходит поиск подпространства в SSI, которому
принадлежит заданная пользователем модель. Затем по геометрическим
параметрам найденного подпространства вычисляются идентификаторы ближайших
к заданной модели.  Полученные идентификаторы – порядковые номера пар из
MI. Т.к. размер всех этих пар равный и фиксированный, то расположение
соответствующих данных в MI вычисляется как произведение идентификатора
модели на размер пары из MI. По полученным идентификаторам сигналов
аналогичным способом происходит поиск соответствующих сигналов в SI.
Результатом решения прямой задачи на основе палеток является сигнал,
полученный аппроксимацией выбранных из SI сигналов (рис. 4).


\subsection{Обратная задача}

Второй тип решаемых задач "--- обратные задачи. Они позволяют по заданным
результатам прямой задачи получить ее ближайшие входные параметры, которые
находятся в палетке. При решении обратной задачи на вход подается сигнал с
прибора, палетка и погрешность измерений. Далее происходит выборка
идентификаторов моделей из SI, соответствующие сигналы которых лежат в
заданной погрешности от заданного сигнала. По полученным идентификаторам
происходит поиск соответствующих идентификаторов подпространств в MI. В SSI
выбирается описание подпространств, соответствующих им идентификаторам. По
идентификатору модели и из описания параметров подпространства получают
параметры соответствующей модели среды. Из этих моделей происходит выборка
представителей. Следующий шаг заключается в выборе представителей из
полученного множества эквивалентных по сигналу моделей ().

Выбор моделей-представителей заключается в том, чтобы разбить множество
полученных моделей на группы (кластеры). Внутри каждой группы должны
оказаться «похожие» объекты, а объекты разных группы должны быть наиболее
различны. Полученная совокупность является результатом решения обратной
задачи на основе палеток.

В качестве метода кластеризации для получения оптимального результата был
выбран алгоритм минимального покрывающего дерева.

Алгоритм минимального покрывающего дерева сначала строит на графе
минимальное покрывающее дерево алгоритмом Краскала, а затем последовательно
удаляет ребра с наибольшим весом. На рис. 6 изображено минимальное
покрывающее дерево, полученное для девяти объектов.

Путём удаления связи, помеченной CD, с длиной равной 6 единицам (ребро с
максимальным расстоянием), получаем два кластера: {A, B, C} и {D, E, F, G,
H, I}. Второй кластер в дальнейшем может быть разделён ещё на два кластера
путём удаления ребра EF, которое имеет длину, равную 4,5 единицам.

           % Глава 2
\chapter{Реализация и применение инструментов для потоковой обработки данных электрокаротажа на основе палеточного подхода} \label{chapt3}


Для решения прямых и обратных задач был разработан программный модуль. При
этом предусмотрена возможность замены хранилища данных без перекомпиляции
приложения, что достигнуто за счет унифицированного интерфейса и библиотек.

Решение прямых задач на основе палеток, как для БК, так и для ВИКИЗ,
выполняется по аналогичному графу. На вход вычислительному модулю подаются
модельные параметры среды и палетка. Затем из имеющихся в палетке моделей
выбираются ближайшие к заданной. По соответствующим сигналам от этих
моделей производится интерполяция. В качестве результата прямой задачи
выдается интерполированное по сигналам значение.


\section{Требования наличия функций обработки данных каротажа} \label{sect3_11}
Основными требуемыми функциями при интерпретации данных ВИКИЗ являются
поправка данных за эксцентриситет, расстановка границ и решение обратной
задачи.

Требуемыми функциями обработки данных БКЗ являются расстановка границ с
учетом вмещающий пород, решение прямой задачи, решение обратной задачи.

\section{Поправка данных за эксцентриситет ВИКИЗ} \label{sect3_12}
Метод ВИКИЗ хорошо зарекомендовал себя в скважинах, пробуренных в Западной
Сибири с применением глинистых буровых растворов. В этих условиях контраст
сопротивлений в среде незначителен. В настоящее время ВИКИЗ все больше
применяется для исследований в скважинах, пробуренных на высокопроводящих
буровых растворах, а также для исследований в разрезах с контрастными по
УЭС горными породами.

При увеличении контраста усиливается влияние геометрической модели среды.
Это приводит к выходу значений кажущегося УЭС за диапазон УЭС в среде.
Классически, при интерпретации данных каротажа применяется модель с зондом
на оси скважины и симметричная цилиндрически слоистая модель среды. А
интерпретация сигналов ВИКИЗ в рамках трехмерных моделей с зондом,
расположенным на стенке скважины, становится практически невозможной из-за
сильного увеличения времени решения прямых задач. Поэтому для интерпретации
данных ВИКИЗ предлагается проводить инверсию в рамках осесимметричной
модели, но перед этим подавлять влияние эксцентриситета на сигналы.

Для подавления влияния эксцентриситета В.С. Игнатовым разработан следующий
алгоритм [13]:
\begin{itemize}
\item На основе трехмерного численного моделирования создаются двухслойные
палетки сигналов при положениях зонда на оси
$\Delta\varphi_{\mbox{\scriptsize о}}$ (синтетический сигнал на оси) и на
стенке $\Delta\varphi_{\mbox{\scriptsize с}}$ (синтетический сигнал на
стенке) скважины от $\rho_{\mbox{\scriptsize п}}$ (УЭС пласта) при
различных $\rho_{\mbox{\scriptsize с}}$ (УЭС скважины) и
$r_{\mbox{\scriptsize с}}$ (радиус скважины).
\item По измеренному на стенке скважины значению
$\Delta\varphi_{\mbox{\scriptsize с}}$ и известным величинам
$\rho_{\mbox{\scriptsize с}}$ и $r_{\mbox{\scriptsize с}}$ по палетке
$\Delta\varphi_{\mbox{\scriptsize c}}(\rho_{\mbox{\scriptsize п}})$
определяется УЭС пласта $\rho_{\mbox{\scriptsize п}}$. Для пластов с
проникновением определенное значение $\rho_{\mbox{\scriptsize п}}$ будет
кажущимся УЭС по модели двухслойной среды «скважина-пласт» (т.е. некоторым
эффективным УЭС среды вне скважины).
\item Затем $\rho_{\mbox{\scriptsize п}}$ по палетке
$\Delta\varphi_{\mbox{\scriptsize c}}(\rho_{\mbox{\scriptsize п}})$ для
известных величин $\rho_{\mbox{\scriptsize с}}$ и 
$r_{\mbox{\scriptsize с}}$ пересчитывается 
в значение $\Delta\varphi_{\mbox{\scriptsize о}}$.
\end{itemize}


Особенностью построения  палетки для данной задачи является малое
количество данных, т.к. время расчета одной трехмерной прямой задачи может
достигать нескольких часов. При этом настройка параметров решателя для
каждой прямой задачи выполняется вручную. Это и обуславливает невозможность
получения сколь угодно плотной расчетной сети для данных палетки.

Расчет данных был разбит на серии по 378 задач. В серии фиксированными
параметрами являются прибор и радиус скважины, причём считается, что зонд
ВИКИЗ расположен на стенке скважины. Зондов в приборе ВИКИЗ "--- 9, а так
же было выбрано 4 радиуса скважины. Итого таких серий было рассчитано 36.
Время исполнения всех задач на одной машине средней производительности
около 70000 часов (это больше 7 лет непрерывной работы). На развёрнутой
нами системе HTCondor расчёты были выполнены за 2 с половиной месяца, при
этом задача считалась только по ночам и в выходные дни [14].

Палетка для процедуры поправки за эксцентриситет строится на основе
решателя прямой задачи, описанном в предыдущей главе. Входными данными
выступают диаметр прибора, радиус скважины, УЭС скважины и УЭС пласта.
Данные (сигнал в секции SI палетки), соответствующие такому набору входных
параметров, "--- сигнал прибора на стенке скважины и соответствующий сигнал
на оси.

Процедура поправки, на основе диаметра прибора, радиуса скважины и УЭС
скважины, определяет все возможные соответствующие УЭС пластов и сигналов,
аппроксимируя по данным в палетке. Затем осуществляется поиск ближайших к
заданному сигналу с прибора на стенке, и интерполяция по соответствующих им
сигналам с оси скважины. Полученное интерполированное значение и будет
результатом поправки за эксцентриситет на основе палетки. Поиск однозначен
в силу природы метода ВИКИЗ, показания прибора монотонны в зависимости от
УЭС пласта двухслойной одномерной модели.


\section{Решение прямой задачи ВИКИЗ}
Для создания решателя прямой задачи ВИКИЗ, ввиду требуемого пространства
моделей и размерности выходного вектора, требуется применить все
инструменты для создания алгоритмов на основе палеток, описанных в
предыдущей главе.

Зачастую, при интерпретации данных ВИКИЗ, значение диэлектрической
проницаемости фиксируют для каждого слоя. Поэтому этот параметр имеет
строго определенное значение. Количество параметров для трехслойной модели
достаточно мало, поэтому расчетная сеть определена так, что она подходит
как для низкоомных (от 0.02 Ом до 0.5 Ом), так и высокоомных (2.0 Ом)
буровых растворов. Наиболее часто радиусы скважины при бурении находятся в
диапазоне от 7 до 20 см. Это учитывается при построении сети. При этом
наиболее часто встречающееся значение "--- это 0.108 м, поэтому сеть
разделена на две пересекающихся (от 0.07 м до 0.108 м и от 0.108 м до 0.2
м). Такие диапазоны параметров для трехслойной модели позволяет оценить
влияние всех параметров УЭС и ширин зон на показания прибора. В таблице 1
описаны основные параметры расчетной сети для трёхслойных моделей ВИКИЗ. По
горизонтали таблица разделена на три части. Каждая из них соответствует
параметрам скважины, ЗП, нетронутой части пласта. По вертикали таблица 1
разбита на четыре подсети. В первой строке описания каждой подсети
обозначено либо значение, либо интервал. На второй "--- мощность интервала.
Внутренние значения в интервале расположены в логарифмическом масштабе.

В таблице 2 заданы параметры четырехслойных моделей ВИКИЗ. В силу
ограничения по памяти у четырехслойных моделей необходимо зафиксировать
несколько параметров. Иначе данных будет так много, что выборка из них
будет происходить существенно дольше, чем это необходимо. Наиболее часто,
исследуемые скважины имеют радиус 0.108 м. Поэтому в этот параметр имеет
фиксированное значение. Дополнительно, палетки построенных на этой сети
ограничены применением только в высокоомных буровых растворах, т.е.
сопротивление скважины имеет фиксированное значение. Таблица 2 имеет
сходную структуру с таблицей 1. Они различаются только наличием параметров
ОЗ.

Расчеты прямых задач ВИКИЗ производились на грид-системе HTCondor,
использующей вычислительные ресурсы ИНГГ СО РАН. Так как время расчета
прямых задач ВИКИЗ составляет порядка 100 мс, то расчетная сеть была
разбита на подсети по 225 тысяч узлов.

\section{Решение прямой задачи БКЗ}
Таблица 3 содержит параметры расчетной сети для трёхслойных моделей БКЗ. В
отличие от таблицы 1 в ней вместо ширины ЗП, задано отношение диаметра ЗП к
диаметру скважины. Это обусловлено программным интерфейсом точного решения
прямой задачи БК. Диапазоны значений сопротивления достаточно велики. Они
охватывают большинство встречающихся в реальных условиях моделей.

\section{Алгоритмы, не требующие реализации палеточных аналогов}
Для того, чтобы библиотека Emfcore имела весь необходимый инструментарий
обработки данных ВИКИЗ и БКЗ, в нее добавлены процедуры расстановки границ
и обратной задачи.

Суть методов расстановки границ заключается в том, чтобы по имеющимся
данным построить кривую, характеризующую возможность наличия границы в
каждой точке. Для имеющихся двух алгоритмов используется разный подход к
формированию этой кривой [15].

Первый это градиентный метод расстановки границ. В своей основе использует
дифференциальные исчисления, взятие производных и выделяет границы в той
области каротажной кривой, которая характеризуется наибольшей скоростью
изменения измеренного параметра. Второй способ расстановки границ "---
дисперсионный метод. В его основе лежат статистические методы определения
границы пласта.

В промышленной обработке данных электрического и электромагнитного каротажа
основным критерием является скорость с достаточной точностью. Поэтому
широкое распространение получили итерационные методы решения обратной
задачи. На основе полученных в ходе зондирования показаний прибора строится
стартовая модель, которая подается на вход прямой задачи для вычисления
теоретических показаний и расчета невязки. Если невязка меньше некоторого
заданного значения, можно считать решение удовлетворительным. Если для
текущей модели теоретические и измеренные показания дают большое
расхождение, модель корректируется и процесс повторяется. Т.е. решение
обратной задачи сводится к многократному решению прямой задачи. Поэтому
использование быстрых прямых задач на основе палеток достаточно ценно.
Решатели обратных задач, которые внедрены в Emfcore, осуществляют
минимизацию функционала невязки методом Нелдера "--- Мида.

Для достоверной оценки фильтрационно-емкостных свойств пласта необходимо
построение как можно более точной геоэлектрической модели. Совместная
инверсия данных по нескольким методам дает возможность построить такую
модель, значительно уменьшив при этом область эквивалентности [16]. В
настоящий момент в библиотеке Emfcore реализован алгоритм решения обратной
задачи, в котором подбор параметров единой модели выполняется сразу по двум
методам.

\section{Количественные характеристики результирующего продукта}

\subsection{Временные характеристики}

Таким образом, как проиллюстрировано в Таблица 4, программная библиотека,
содержащая быстрые алгоритмы для интерпретации, позволяет существенно
сократить время обработки при потоковой обработке данных БКЗ и ВИКИЗ.

Все алгоритмы реализовывались так, чтобы погрешность измерений была меньше
погрешности измерений приборами ЗАО НПП ГА «Луч» для соответствующих
методов.

Решением обратной задачи является множество моделей, сигналы которых
отличаются от искомого в заданных пользователем пределах. В следствии
неединственности решения обратной задачи, в пространстве моделей существуют
области эквивалентных по сигналу моделей, что является следствием как
модельно обусловленной эквивалентностью для метода ВИКИЗ, так и
погрешностями измерений. В результате испытаний было выявлено, что прямая
выборка подходящих моделей из хранилища данных может возвращать более 1000
элементов. При этом,  только небольшая часть из них существенно различаются
друг от друга по модельным параметрам, концентрируясь вокруг
моделей-представителей. В связи с этим, была реализована предобработка
результатов выборки. Для выбора моделей-представителей множество полученных
моделей разбивается на заданное количество кластеров по принципу схожести.
Внутри каждого кластера должны оказаться «похожие» объекты, а объекты
разных кластеров должны быть максимально различаться. Из каждого кластера
выбирается модель-представитель (рис. 5). Полученная совокупность моделей
является результатом решения обратной задачи на основе палеток.

Вычисление точного решения прямой задачи ВИКИЗ производилось алгоритмом
ALVIK [2]. Его средняя скорость решения на обычном современном ПК
(производительность порядка 0.1 Терафлопс) составляет 60 мс для трехслойных
моделей и 90 мс для четырехслойных. Средняя скорость решения с применением
палеток около 1 мс для трех- и четырехслойных моделей. Обратная задача
ВИКИЗ выдает результат спустя 0.1--2 с после запуска, в зависимости от
размера области эквивалентных решений. Для БК эти показатели на два порядка
лучше. Это обусловлено низкой скоростью решения исходной реализацией прямой
задачи.

В связи с полученными временными характеристиками палеточного подхода для
методов электрического каротажа, при совместной интерпретации комплекса
электрометрии в скважинах скорость решения задачи БК будет критическим
фактором. Таким образом, использование палеток при совместной интерпретации
позволит существенно увеличить её производительность работы, а проведение
анализа областей эквивалентности повысит качество заключений.


\section{Варианты реализации: файлы, БД, код} \label{sect3_1}


\section{Описание инструментов для решения задач геоэлектрики} \label{sect3_2}

\section{Примеры решения практических задач: исследовательских, промышленных} \label{sect3_3}


\clearpage           % Глава 3
\chapter{Программная система на принципах потоковой работы с разнородными данными} \label{chapt4}

\section{Требования к компьютерной системе} \label{sect4_1}

Существует ряд проблем, связанных с тем, что реальная модель среды более
сложная и при интерпретации существующими методиками получаются не совсем
корректные результаты. Так помимо того, что следует комплексировать методы
геоэлектрики, также стоит учитывать региональное напряженно деформированное
состояние и напряжение, возникающие в околоскважинном пространстве в
процессе бурения. Так авторы статьи~\cite{Yeltsov2014} Ельцов, Соболев,
Назаровы, Нестерова и другие, показали, что неоднородное
напряженно-деформированное состояние разбуриваемого пласта приводит к
существенному изменению пространственного распределения водонасыщенности и
солености, что в свою очередь обуславливает сложное распределение
электрических параметров, получаемых при электрометрии в скважине. Если не
учитывать этот факт, то пористость и проницаемость может быть оценена не
верно.

В работе предлагается разработать программное средство, которое бы
позволяло строить междисциплинарную модель прискважинной области,
включающую геомеханику, гидродинамику и электрометрию.

Таким образом в ПС должно включать теоретико-экспериментальную модель
сопровождения бурения и исследования в скважинах, обеспечивающую
диагностику состояния и определение свойств околоскважинного пространства и
нефтяного резервуара.

Это накладывает особые требования, которые и будут отличать данное ПС от
существующих систем. Это:
\begin{itemize}
\item оперировать разнородными входными данными, это как данные по бурению,
так и каротажные данные с учётом времени их получения
\item возможность интеграции сложных объектов как на уровне вычислительных
модулей, так и на уровне пользовательского интерфейса.
\end{itemize}

Такие требования позволят как проводить исследования в построении
междисциплинарных моделей, так и оперативно внедрять эти разработки в
промышленное использование, учитывая, что ПС будет иметь критически важные
в прикладном деле инструменты, такие как расчёт показателей стабильности
ствола скважины в сопровождении бурения, интерпретации электрометрии и
вычисления петрофизических параметров для оценки запасов.

На основе алгоритмов, разработанных в рамках данной работы, была создана
удовлетворяющая таким требованиям компьютерная система для потоковой
обработки разнородных данных.

\section{Принципы построение пользовательского интерфейса}

\section{Организация данных проекта}

\section{Модульная архитектура}

\clearpage           % Глава 4
\include{Dissertation/conclusion}      % Заключение
\include{Dissertation/acronyms}        % Список сокращений и условных обозначений
\include{Dissertation/dictionary}      % Словарь терминов
\include{Dissertation/references}      % Список литературы
\include{Dissertation/lists}           % Списки таблиц и изображений (иллюстративный материал)

%%% Настройки для приложений
\appendix
% Оформление заголовков приложений ближе к ГОСТ:
\setlength{\midchapskip}{20pt}
\renewcommand*{\afterchapternum}{\par\nobreak\vskip \midchapskip}
\renewcommand\thechapter{\Asbuk{chapter}} % Чтобы приложения русскими буквами нумеровались

\chapter{Примеры вставки листингов программного кода} \label{AppendixA}

Для крупных листингов есть два способа. Первый красивый, но в нём могут быть
проблемы с поддержкой кириллицы (у вас может встречаться в~комментариях
и печатаемых сообщениях), он представлен на листинге~\ref{list:hwbeauty}.
\begin{ListingEnv}[!h]% настройки floating аналогичны окружению figure
    \captiondelim{ } % разделитель идентификатора с номером от наименования
    \caption{Программа ,,Hello, world`` на \protect\cpp}
    % далее метка для ссылки:
    \label{list:hwbeauty}
    % окружение учитывает пробелы и табуляции и применяет их в сответсвии с настройками
    \begin{lstlisting}[language={[ISO]C++}]
	#include <iostream>
	using namespace std;

	int main() //кириллица в комментариях при xelatex и lualatex имеет проблемы с пробелами
	{
		cout << "Hello, world" << endl; //latin letters in commentaries
		system("pause");
		return 0;
	}
    \end{lstlisting}
\end{ListingEnv}%
Второй не~такой красивый, но без ограничений (см.~листинг~\ref{list:hwplain}).
\begin{ListingEnv}[!h]
    \captiondelim{ } % разделитель идентификатора с номером от наименования
    \caption{Программа ,,Hello, world`` без подсветки}
    \label{list:hwplain}
    \begin{Verb}

        #include <iostream>
        using namespace std;
        
        int main() //кириллица в комментариях
        {
            cout << "Привет, мир" << endl;
        }
    \end{Verb}
\end{ListingEnv}

Можно использовать первый для вставки небольших фрагментов
внутри текста, а второй для вставки полного
кода в приложении, если таковое имеется.

Если нужно вставить совсем короткий пример кода (одна или две строки),
то~выделение  линейками и нумерация может смотреться чересчур громоздко.
В таких случаях можно использовать окружения \texttt{lstlisting} или
\texttt{Verb} без \texttt{ListingEnv}. Приведём такой пример
с указанием языка программирования, отличного от~заданного по умолчанию:
\begin{lstlisting}[language=Haskell]
fibs = 0 : 1 : zipWith (+) fibs (tail fibs)
\end{lstlisting}
Такое решение~--- со вставкой нумерованных листингов покрупнее
и вставок без выделения для маленьких фрагментов~--- выбрано,
например, в книге Эндрю Таненбаума и Тодда Остина по архитектуре
%компьютера~\autocite{TanAus2013} (см.~рис.~\ref{fig:tan-aus}).

Наконец, для оформления идентификаторов внутри строк
(функция \lstinline{main} и~тому подобное) используется
\texttt{lstinline} или, самое простое, моноширинный текст
(\texttt{\textbackslash texttt}).

Пример~\ref{list:internal3}, иллюстрирующий подключение переопределённого
языка. Может быть полезным, если подсветка кода работает криво. Без
дополнительного окружения, с подписью и ссылкой, реализованной встроенным
средством.
\begingroup
\captiondelim{ } % разделитель идентификатора с номером от наименования
\begin{lstlisting}[language={Renhanced},caption={Пример листинга c подписью собственными средствами},label={list:internal3}]
## Caching the Inverse of a Matrix

## Matrix inversion is usually a costly computation and there may be some
## benefit to caching the inverse of a matrix rather than compute it repeatedly
## This is a pair of functions that cache the inverse of a matrix.

## makeCacheMatrix creates a special "matrix" object that can cache its inverse

makeCacheMatrix <- function(x = matrix()) {#кириллица в комментариях при xelatex и lualatex имеет проблемы с пробелами
    i <- NULL
    set <- function(y) {
        x <<- y
        i <<- NULL
    }
    get <- function() x
    setSolved <- function(solve) i <<- solve
    getSolved <- function() i
    list(set = set, get = get,
    setSolved = setSolved,
    getSolved = getSolved)
    
}


## cacheSolve computes the inverse of the special "matrix" returned by
## makeCacheMatrix above. If the inverse has already been calculated (and the
## matrix has not changed), then the cachesolve should retrieve the inverse from
## the cache.

cacheSolve <- function(x, ...) {
    ## Return a matrix that is the inverse of 'x'
    i <- x$getSolved()
    if(!is.null(i)) {
        message("getting cached data")
        return(i)
    }
    data <- x$get()
    i <- solve(data, ...)
    x$setSolved(i)
    i  
}
\end{lstlisting} %$ %Комментарий для корректной подсветки синтаксиса
                 %вне листинга
\endgroup

Листинг~\ref{list:external1} подгружается из внешнего файла. Приходится
загружать без окружения дополнительного. Иначе по страницам не переносится.
\begingroup
\captiondelim{ } % разделитель идентификатора с номером от наименования
    \lstinputlisting[lastline=78,language={R},caption={Листинг из внешнего файла},label={list:external1}]{listings/run_analysis.R}
\endgroup


\section{Очередной подраздел приложения} \label{AppendixB4}

Нужно больше подразделов приложения!

\section{И ещё один подраздел приложения} \label{AppendixB5}

Нужно больше подразделов приложения!
        % Приложения

\end{document}
