\chapter{Программная система на принципах потоковой работы с разнородными данными} \label{chapt4}

\section{Требования к компьютерной системе} \label{sect4_1}

Существует ряд проблем, связанных с тем, что реальная модель среды более
сложная и при интерпретации существующими методиками получаются не совсем
корректные результаты. Так помимо того, что следует комплексировать методы
геоэлектрики, также стоит учитывать региональное напряженно деформированное
состояние и напряжение, возникающие в околоскважинном пространстве в
процессе бурения. Так авторы статьи~\cite{Yeltsov2014} Ельцов, Соболев,
Назаровы, Нестерова и другие, показали, что неоднородное
напряженно-деформированное состояние разбуриваемого пласта приводит к
существенному изменению пространственного распределения водонасыщенности и
солености, что в свою очередь обуславливает сложное распределение
электрических параметров, получаемых при электрометрии в скважине. Если не
учитывать этот факт, то пористость и проницаемость может быть оценена не
верно.

В работе предлагается разработать программное средство, которое бы
позволяло строить междисциплинарную модель прискважинной области,
включающую геомеханику, гидродинамику и электрометрию.

Таким образом в ПС должно включать теоретико-экспериментальную модель
сопровождения бурения и исследования в скважинах, обеспечивающую
диагностику состояния и определение свойств околоскважинного пространства и
нефтяного резервуара.

Это накладывает особые требования, которые и будут отличать данное ПС от
существующих систем. Это:
\begin{itemize}
\item оперировать разнородными входными данными, это как данные по бурению,
так и каротажные данные с учётом времени их получения
\item возможность интеграции сложных объектов как на уровне вычислительных
модулей, так и на уровне пользовательского интерфейса.
\end{itemize}

Такие требования позволят как проводить исследования в построении
междисциплинарных моделей, так и оперативно внедрять эти разработки в
промышленное использование, учитывая, что ПС будет иметь критически важные
в прикладном деле инструменты, такие как расчёт показателей стабильности
ствола скважины в сопровождении бурения, интерпретации электрометрии и
вычисления петрофизических параметров для оценки запасов.

На основе алгоритмов, разработанных в рамках данной работы, была создана
удовлетворяющая таким требованиям компьютерная система для потоковой
обработки разнородных данных.

\section{Принципы построение пользовательского интерфейса}

\section{Организация данных проекта}

\section{Модульная архитектура}

\clearpage