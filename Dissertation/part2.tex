\chapter{Палеточный подход} \label{chapt2}

В статье предлагается схема формирования и использования палеток для
различных задач одномерного моделирования, используя современные
вычислительные мощности. Это позволяет обойти упомянутые выше недостатки
существующих реализаций палеточного подхода. В данной работе
рассматриваются прямые и обратные задачи ВИКИЗ и БК на одномерных
многопараметричных моделях, что, впрочем, не ограничивает общности подхода.
Такие модели представлены набором вложенных цилиндрических колец (слоёв) с
бесконечной шириной внешнего слоя (рис.1). Каждый слой имеет однородное
пространственное распределение относительной диэлектрической проницаемости
(ОДП) и удельного электрического сопротивления (УЭС).

В общем виде схема формирования палеток состоит из пяти взаимосвязанных
процедур (этапов), образующих итеративный процесс (рис. 2).


\section{Выбор расчетной сетки} \label{sect2_1}

Целью предварительного расчета прямых задач является ускорение за счет
быстрого поиска. Наиболее быстро поиск осуществляется при регулярно
организованных данных. Скорость поиска по таким данным мало зависит от
количества данных, а сам поиск сводится к выборке по конкретному адресу.
Это возможно осуществить, если закодировать входные данные одним числом
"--- порядковым номером в конечномерной расчетной сети.

Одной из главных характеристик расчетной сети является величина шага.
Исходя из природы электромагнитных методов, целесообразно рассматривать
регулярную сеть с экспоненциальным шагом узлов.

Диапазон значений параметров среды желательно выбирать как можно шире.
Число узлов в заданной расчетной сети зависит от имеющих вычислительных
ресурсов и скорости вычисления точного решения прямой задачи. Следует
учитывать, что есть узлы, для которых желательно получать более точные
решения с высокой скоростью. Для этого расчетную сеть можно разделить  на
несколько пересекающихся по нужным параметрам.

\section{Проведение расчета на высокопроизводительной системе} \label{sect2_2}

Для случая, когда время вычисления задач на одной вычислительной машине для
заданной расчетной сети неприемлемо велико, разработан инструмент для
разбиения всего множества задач на подзадачи заданного размера. Это
необходимо для автоматизированного расчета на множестве вычислительных
узлов. Такими узлами могут выступать элементы грид-системы [10].
Преимущество такой системы в том, что она идеально подходит для расчета
задач на сетке. При этом на программу точного (аналитического) решения не
накладывается особых условий. Тем не менее, желательно, чтобы за один
запуск программы рассчитывалась не один узел сети, а некоторое её
подмножество. Это условие происходит из того, что время на передачу данных
должно быть меньше времени расчета на одной машине.

Преимуществом грид-систем перед кластерами в возможности запускать
имеющиеся бинарные программы для ОС Windows на рабочих станциях без
перекомпиляции кода.

В качестве грид-системы был выбран HTCondor [11]. Его преимущества
заключается в том, что он бесплатный, достаточно часто обновляется и
работает как на Unix подобных ОС, так и на ОС семейства Windows.

\section{Оптимизация сетки} \label{sect2_3}

Преимущества регулярной сети могут частично нивелироваться в случае, если
количество выходных данных слишком велико. В частности, количество
измеряемых данных высокочастотными индукционными приборами может достигать
двух десятков штук на каждый узел сети. Размер результирующих данных для
сети, покрывающей рассматриваемую область входных параметров с достаточной
точностью при аппроксимации по ним, составляет около четырех гигабайт. Это
означает, что время позиционирования по данным будет больше, чем получение
результата имеющимся решателем.

Исходя из вышеизложенного, был разработан оптимизирующий данные модуль, в
задачи которого входят уменьшение количества данных, при сохранении
локальной регулярности и возможности аппроксимации по этим данным с
требуемой точностью.

Суть лежащего в основе инструмента алгоритма заключается в удалении узлов
исходной сети по следующему правилу. Считается, что узел можно удалить,
если смоделированный в нём сигнал интерполируется сигналами соседних узлов
и разница между ними находится в пределах заданной погрешности. При этом
допускается погрешность равная погрешности измерений прибора. Интерполяция
проводится полиномом второй степени, построенным через три ближайшие точки.
В процессе оптимизации, узлы удаляются таким образом, что сеть оказывалась
разбитой на части, далее зоны, имеющие регулярную структуру. Такой подход
позволяет: во-первых, сокращать объём сети, сохраняя быстрый доступ к её
узлам благодаря регулярности зон, и, во-вторых, допускает уточнение
поведения сигналов на некоторой области моделей, путём пересчёта зоны, в
которую входит интересующая нас область, с более мелким шагом.

На рисунках ниже (рис. 1, рис. 2) представлена программная реализация
модуля оптимизации насчитанных на грид-системе данных. Данные находятся в
нескольких файлах, по числу подзадач, на которые разбита расчетная сеть. В
каждом файле результат расчета прямой задачи на некотором связанном
подмножестве моделей (т.е. на n-мерном параллелепипеде, где n-количество
параметров модели среды). Для работы с результатами расчетов, как с единым
пространством служит класс GridDataController. Класс GridZone хранит
информацию о пространстве моделей: геометрию и значения параметров модели
среды. Класс, ответственный за оптимизацию, "--- GirdOptimizeController.

Отношения рассмотренных трех сущностей [12] (GridDataController, GridZone,
GirdOptimizeController) составляют основу модуля по оптимизации насчитанных
данных (рис. 1).

Ниже представлена функциональная схема модуля по оптимизации насчитанных
данных (рис. 2).


\section{Организация в единое хранилище в зависимости от решаемой задачи} \label{sect2_4}

Для организации полученных результатов были разработаны форматы
представления для двух способов хранения: в бинарном файле и в базе данных.
Во втором случае была использована база данных SQLite в виду ряда важных
преимуществ, таких как:
\begin{itemize}
\item она свободна;
\item API доступен в виде отдельного С файла без внешних зависимостей;
\item поддержка хранения больших объёмов данных: до терабайта; 
\item относительно быстро происходит выборка данных из таблиц;
\item отсутствие сервера при работе с базой данных.
\end{itemize}

Плюсы использования базы данных в качестве хранилища – простота изменения и
добавления данных.  Хранение данных непосредственно в бинарном файле
уменьшает время выборки и объём, но теряется гибкость работы с содержимым.

Программный модуль работы с базой данных включает функции создания,
добавления и удаления регулярных подпространств. На рисунке (рис. 3)
приведена структура базы данных SQLite этого модуля, т.е. таблицы и связи
между ними. Таблицы проектировались таким образом, чтобы такие операции как
соединение, добавление и удаление подобластей, на которые разбиты данные,
выполнялись быстро.

Таблица Properties содержит информацию о свойствах палеток, таких как
точность представления данных, идентификатор алгоритма решателя, параметры
прибора.  Модель представляется одним числом (поле id в таблице Model).
Есть отдельная таблица (SubSpace) для хранения подпространств рассчитанных
данных. Т.к. подпространство – это многомерный параллелепипед, то в таблице
хранятся только координаты двух его вершин. Для нахождения моделей близких
к заданной в таблице SubSpace хранятся начальный (id\_b) и конечный (id\_e)
идентификаторы моделей. Каждый сигнал имеет свой уникальный номер, который
хранится в поле id таблицы Signal. Для увеличения скорости нахождения
сигналов, близких к заданному, существует дополнительная таблица
Signal\_marks.

Модуль организации данных палетки в бинарном файле создает три секции,
которые содержат:
\begin{itemize}
\item информацию о подпространствах (секция SSI);
\item информацию о связи моделей и подпространств (MI);
\item информацию о сигналах и их связях с моделями (SI).
\end{itemize}

Секция SSI состоит из совокупности векторов. Эти вектора содержат
уникальный идентификатор подпространства и информацию об этом
подпространстве. Эти вектора упорядочены по идентификатору в
лексикографическом порядке.

Идентификатор сигнала равен номеру его вхождения в SI. Секция MI состоит из
множества пар, которые содержат идентификатор подпространства из SSI и
идентификатор сигналов. Идентификатор модели соответствует порядку
расположения этих пар в MI. Секция SI в свою очередь состоит и пар "--- 
идентификатор модели и сигнал.

\section{Поиск решений и аппроксимации по данным палетки} \label{sect2_5}

Разработанный модуль поиска может решать две задачи.

\subsection{Прямая задача}

Первая задача заключается в нахождении ближайших данных, соответствующих
заданной модели, аппроксимации по ним. Такую задачу назовем прямой задачей.
При решении прямой задачи происходит поиск подпространства в SSI, которому
принадлежит заданная пользователем модель. Затем по геометрическим
параметрам найденного подпространства вычисляются идентификаторы ближайших
к заданной модели.  Полученные идентификаторы – порядковые номера пар из
MI. Т.к. размер всех этих пар равный и фиксированный, то расположение
соответствующих данных в MI вычисляется как произведение идентификатора
модели на размер пары из MI. По полученным идентификаторам сигналов
аналогичным способом происходит поиск соответствующих сигналов в SI.
Результатом решения прямой задачи на основе палеток является сигнал,
полученный аппроксимацией выбранных из SI сигналов (рис. 4).


\subsection{Обратная задача}

Второй тип решаемых задач "--- обратные задачи. Они позволяют по заданным
результатам прямой задачи получить ее ближайшие входные параметры, которые
находятся в палетке. При решении обратной задачи на вход подается сигнал с
прибора, палетка и погрешность измерений. Далее происходит выборка
идентификаторов моделей из SI, соответствующие сигналы которых лежат в
заданной погрешности от заданного сигнала. По полученным идентификаторам
происходит поиск соответствующих идентификаторов подпространств в MI. В SSI
выбирается описание подпространств, соответствующих им идентификаторам. По
идентификатору модели и из описания параметров подпространства получают
параметры соответствующей модели среды. Из этих моделей происходит выборка
представителей. Следующий шаг заключается в выборе представителей из
полученного множества эквивалентных по сигналу моделей ().

Выбор моделей-представителей заключается в том, чтобы разбить множество
полученных моделей на группы (кластеры). Внутри каждой группы должны
оказаться «похожие» объекты, а объекты разных группы должны быть наиболее
различны. Полученная совокупность является результатом решения обратной
задачи на основе палеток.

В качестве метода кластеризации для получения оптимального результата был
выбран алгоритм минимального покрывающего дерева.

Алгоритм минимального покрывающего дерева сначала строит на графе
минимальное покрывающее дерево алгоритмом Краскала, а затем последовательно
удаляет ребра с наибольшим весом. На рис. 6 изображено минимальное
покрывающее дерево, полученное для девяти объектов.

Путём удаления связи, помеченной CD, с длиной равной 6 единицам (ребро с
максимальным расстоянием), получаем два кластера: {A, B, C} и {D, E, F, G,
H, I}. Второй кластер в дальнейшем может быть разделён ещё на два кластера
путём удаления ребра EF, которое имеет длину, равную 4,5 единицам.

