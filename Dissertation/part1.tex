\chapter{Обзор существующих методик решения практических и теоретических решений прямых и обратных задач скважинной геофизики} \label{chapt1}

Тут я процитирую работы чтобы появились в списке литературы~\cite{Serdyuk2011,Serdyuk2014,Serdyuk2017,Baranova2014,Koshelev2012,Serdyuk2013a,Teit2011,Uramaev2015}.

Для компенсации избыточного давления стенок скважины в процессе бурения и
проведения каротажа используют буровой раствор. Он также применяется для
промывки скважины от бурового шлама. Буровые растворы могут быть сделаны на
водной, углеводной и аэрированной основах [5]. Выбор основы зависит от типа
бурящихся пород. Фильтрат бурового раствора проникает в породу, образуя
зоны с различными УЭС и ОДП. Количество таких зон зависит от фильтрационных
свойств среды.

Структура пластов может быть достаточно неоднородной, и для ее описания
потребовалось бы большое количество параметров. Поэтому вместо среды
используют приближённую модель, которая описывает основные свойства с точки
зрения рассматриваемого метода исследования.

При интерпретации данных электрических и электромагнитных методов в каждой
точке измерений среда моделируется набором радиальных слоев с
концентрическими цилиндрическими границами и с постоянными значениями
параметров внутри каждого слоя. Для метода ВИКИЗ и БК искомые параметры
"--- УЭС и ширины слоёв. Эти параметры являются ключевыми, но кроме того,
на показания зондов ВИКИЗ в некоторых случаях влияет ОДП.

Наиболее приближенной к реальной описывают среду трех- и четырехслойные
модели.  Трехслойная модель (рис. 1) состоит из скважины, зоны
проникновения (ЗП) и нетронутой буровым раствором части пласта.
Четырехслойная модель (рис. 2) описывает ситуацию, когда проникший в породу
фильтрат бурового раствора образует зону проникновения и окаймляющую зону
(ОЗ).

Для задач скважинной геомеханики основная задача состоит в определении
механических свойств породы и нахождения устойчивого состояния стенок
скважины. Как правило модели предполагают наличие частичной симметричности
относительно оси скважины, но не учитывают наличие радиальных слоев помимо
самой скважины. Для таких задач обычно встает вопрос поиска корреляционных
зависимостей между данными исследований образцов КЕРНа и данными ГИС.

В процессе интерпретации каротажа происходит решение прямых и обратных
задач. Прямая задача заключается в определении показаний прибора,
находящегося в заданной модели среды. Обратная же задача, наоборот, состоит
в определении среды, в которой величины измерений прибора совпадают с
заданными.

\section{Походы к решению задач: палетки, оптимизационные методы, нейросетевые} \label{sect1_1}

\subsection{Аналитические решатели} \label{subsect1_1_1}
Для электрических и электромагнитных методов существуют алгоритмы точного
решения прямых задач. Они основаны на законах Максвелла. Такие алгоритмы
зачастую работают медленно.

Для ВИКИЗ, одним из точных алгоритмов является ALVIK [7], разработанный и
реализованный в ИГиГ СО АН СССР. Особенность данного алгоритма состоит в
том, что он позволяет решать прямые задачи для произвольных конфигураций
прибора типа ВИКИЗ. Скорость решения одной одномерной трёхслойной задачи
составляет 90 мс, а четырёхслойной "---  120 мс.

Представителем реализованных алгоритмов точного решения для бокового
каротажа является реализации Миронцова, сотрудника ИНГГ СО РАН. Этот
алгоритм был разработан специально для моделирования прибора,
разработанного в ЗАО НПП ГА «Луч». Он предназначен для решения трёхслойных
одномерных задач. Скорость решения такой прямой задачи на порядок ниже, чем
у ВИКИЗ "--- порядка двух секунд.

\subsection{Классический палеточный подход} \label{subsect1_1_2}
Решение обратной задачи на основе палеток осуществлялось еще до широкого
распространения ЭВМ. Сперва насчитывались прямые задачи и по результатам
строились графики зависимости сигнала прибора от сопротивления однородной
среды. Такие  графики наносились на кальку. Решение обратной задачи
сводилось к наложению кальки к каротажным кривым. Задача считалась
решенной, если кривые совпадали с некоторой точностью. Несмотря на
архаичность данного метода, он до сих пор используется в некоторых
нефтедобывающих компаниях.

Более современные реализации палеток – совокупность заранее насчитанных
прямых задач и их входных данных. Решение обратной задачи на таких палетках
сводится к выборке входных данных прямой задачи по результатам её решения.
Палетками также может служить нейронная сеть, обученная на множестве
решений прямых задач [8].

Существующие реализации решения обратных задач на основе палеток работают с
двух- или трёхслойными одномерными моделями. Причем трёхслойные модели
имеют фиксированные значения некоторых параметров. Это определяется
необходимостью иметь большое количество памяти и вычислительных ресурсов.
Так же можно отметить, что область покрытия существующими палетками
необходимых моделей недостаточна.

\subsection{Нейросети} \label{subsect1_1_3}

\subsection{Оптимизационные алгоритмы} \label{subsect1_1_4}

Другой комплекс методов решения обратной задачи "--- это итерационные
методы. Схема решения  представлена на рис. 5. На основе полученных в ходе
зондирования показаний прибора строится стартовая модель, которая подается
на вход прямой задачи для вычисления теоретических показаний и расчета
невязки. Если невязка меньше некоторого заданного значения, можно считать
решение удовлетворительным. Если для текущей модели теоретические и
измеренные показания дают большое расхождение, модель корректируется и
процесс повторяется. Т.е. решение обратной задачи сводится к многократному
решению прямой задачи. Поэтому критично, чтобы скорость решения прямой
задачи была как можно выше.


\section{Существующее ПО для интерпретации} \label{sect1_2}

Для интерпретации данных электромагнитного каротажа большинство российских
компаний используют лишь небольшое количество различных комплексных
программных систем. Среди таких систем Techlog [4] от компании
Schlumberger, СИАЛ-ГИС [5], ПРАЙМ [6], «ГеоПоиск» [7], Geo Office Solver
[8]. Другие продукты не рассматриваются ввиду малой распространенности на
территории Западной Сибири.

Рассмотрим функционал представленных программных систем с точки зрения
средств визуализации, хранения и обработки данных электрометрических
исследований в скважинах.

\subsection{ПРАЙМ} \label{subsect_Prime}
Программная система ПРАЙМ является автоматизированным инструментом в среде
Windows для сбора, визуализации, обработки и хранения геолого-геофизических
данных в нефтегазовой отрасли. Система имеет модульную структуру, которая
позволяет легко собирать рабочие места различной сложности и назначения для
удовлетворения потребностей конкретного пользователя. Такая структура
позволяет обеспечить масштабируемость от мелких компаний до крупных
корпораций.

Организация данных и ее функциональные возможности позволяют решать задачи
сбора, анализа и обработки данных ГИС на всех этапах жизни скважины:
открытого ствола, цементирования и обсаженного ствола.

ПРАЙМ является не только готовой системой для практического использования,
но и универсальной средой для разработки новых приложений. Открытость для
расширения и развития в стандарте DLL, COM и внутреннего языка позволяют
пользователям самостоятельно развивать возможности системы для адаптации к
конкретным условиям. Расширения системы ПРАЙМ представляют собой модули
(DLL), которые могут быть подключены к системе в любой момент без
привлечения разработчиков системы. Допускаются любые сочетания расширений.
Каждое расширение системы ПРАЙМ обычно позволяют решить отдельную задачу
обработки данных ГИС.

Модули системы Прайм условно разделены по блокам. Сервер данных и сервер
планшета обеспечивают разработчиков около 1000 универсальными готовыми
функциями для разработки приложений.

Оригинальная технология построения и управления локальной базой данных
обеспечивает необходимую универсальность и гибкость при импорте-экспорте и
построении технологий обработки данных на местах. Отличается от других
систем тем, что не навязывает собственную модель данных или внутренний
формат представления каротажных кривых.

Поддерживает как непрерывные, так и попластовые типы данных и технологии их
обработки, в том числе и для смешанных типов данных. Имеются специальные
средства для работы с данными керна. Пользователь может самостоятельно
создавать или пользоваться готовыми графами обработки данных.

\subsection{ГеоПоиск} \label{subsect_Poisk}
ГеоПоиск "--- это программный пакет в среде Windows, предназначенный для
обработки и интерпретации данных геофизического исследования скважин с
привлечением смежной информации на уровнях от отдельной скважины, куста
скважин до месторождения и группы месторождений.

Система имеет собственную общую структурированную геофизическую базу
данных. Все коммуникации между различными модулями системы осуществляются
через эту БД. Существует возможность потоковой обработки множества скважин.
Для этого в программный комплекс встроена следующая функциональность:
потоковая загрузка в базу с автоматическим созданием и заполнением скважин
данными; выполнение алгоритмов (формул) для группы выбранных скважин;
создание текстовых отчетов-таблиц любой сложности по данным группы скважин,
месторождению; «однокнопочное» формирование планшета скважины по шаблону
любой сложности. Для визуализации данных используются подход, основанный на
макетном (шаблонном) представлении.

\subsection{GeoOffice Solver} \label{subsect_Solver}
Программный комплекс GeoOffice Solver предназначено для сбора,
архивирования и интерпретации данных ГИС. Он включает в себя все
необходимые вычислительные алгоритмы для предобработки и интерпретации
данных БКЗ, визуализирует полученную геоэлектрическую модель. Для обработки
данных ВИКИЗ, система GeoOffice Solver интегрирована с комплексом МФС
ВИКИЗ.

\subsection{СИАЛ-ГИС} \label{subsect_Sial}
СИАЛ-ГИС "--- автоматизированная система обработки и интерпретации данных
каротажа скважин, ориентированная на промышленную эксплуатацию как в
геофизических предприятиях для оперативной интерпретации и выдачи
окончательного заключения по скважинам, так и в научно-исследовательских и
проектных институтах для интерпретации материалов ГИС при пересчете запасов
и построении постоянно действующих геологических и гидродинамических
моделей месторождений.

Система реализует в полном объеме все этапы традиционной интерпретации
данных каротажа. Алгоритмы системы являются результатом формализации
классических приемов и способов, пригодных для любого района. Кроме того,
система содержит решения специфических задач, актуальных в Западной Сибири,
с учетом специфики существующего ограниченного комплекса измерений в
эксплуатационных скважинах и геолого-геофизических особенностей района
работ.

Система имеет, в качестве основного, вариант полностью непрерывного
автоматизированного процесса интерпретации по скважине от загрузки
каротажных кривых на входе, до формирования заключения на выходе,
содержащего полную информацию о литологии пластов, о наличии коллекторов в
обрабатываемом разрезе, о характере и степени их насыщения,
фильтрационно-емкостных свойствах. В то же время система предоставляет
широкие возможности контроля и корректировки исходных данных, промежуточных
и окончательных результатов в интерактивно-графическом режиме, позволяющие
интерпретатору по желанию вмешиваться в процесс обработки, принимая в ней
творческое участие на наиболее ответственных этапах.

Обработка данных электрометрии по ВИКИЗ построена с использованием модуля
МФС ВИКИЗ, поставляемого ИНГГ СО РАН. Стыковка произведена по следующей
схеме: в модуль МФС из СИАЛ-ГИС подаются литологические границы. В самом
МФС ВИКИЗе производится автоматическое снятие отсчетов по алгоритмам,
принятым для ВИКИЗ, добавляются омические границы в интерактивном режиме и
производится определение Рп обычным для этой программы способом. По
завершении этой работы пользователь просто выходит из модуля ВИКИЗ и
попадает обратно в СИАЛ-ГИС. Вся информация, полученная в ВИКИЗ,
сохраняется в СИАЛ-ГИС и реэкспортируется в ВИКИЗ при желании пересмотреть
или уточнить Рп по ВИКИЗ.

\subsection{Techlog} \label{subsect_Techlog}
Techlog "--- крупный программный геофизический комплекс, разработанный
компанией Schlumberger. Он предназначен для обработки и интерпретации
каротажных данных. В спектр его сервисов входит импорт LAS-файлов, хранение
каротажных кривых, визуализация данных, набор математических алгоритмов
трансформаций и др. Данный продукт позволяет создавать собственные
алгоритмы и интегрировать сторонние модули, что позволяет расширить набор
предлагаемых возможностей. Механизм интеграции представлен в виде
исполняемых скриптов, написанных на языке Python. В Techlog встроен
собственный Python редактор с отладчиком. Программные модули встраиваются в
панель инструментов в виде меню, написанных на языке xml. Как было отмечено
ранее, отличительной особенностью Techlog является предоставление
возможности автоматической интерпретации данных. Пользователь может создать
так называемый Python Workflow Item (Python AWI), который объединит
необходимые скрипты в исполняемый непрерывно конвейер. На вход такого
процесса могут быть подано сразу множество однотипных данных, полученных в
результате исследований в различных скважинах.

