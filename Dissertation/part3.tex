\chapter{Реализация и применение инструментов для потоковой обработки данных электрокаротажа на основе палеточного подхода} \label{chapt3}


Для решения прямых и обратных задач был разработан программный модуль. При
этом предусмотрена возможность замены хранилища данных без перекомпиляции
приложения, что достигнуто за счет унифицированного интерфейса и библиотек.

Решение прямых задач на основе палеток, как для БК, так и для ВИКИЗ,
выполняется по аналогичному графу. На вход вычислительному модулю подаются
модельные параметры среды и палетка. Затем из имеющихся в палетке моделей
выбираются ближайшие к заданной. По соответствующим сигналам от этих
моделей производится интерполяция. В качестве результата прямой задачи
выдается интерполированное по сигналам значение.


\section{Требования наличия функций обработки данных каротажа} \label{sect3_11}
Основными требуемыми функциями при интерпретации данных ВИКИЗ являются
поправка данных за эксцентриситет, расстановка границ и решение обратной
задачи.

Требуемыми функциями обработки данных БКЗ являются расстановка границ с
учетом вмещающий пород, решение прямой задачи, решение обратной задачи.

\section{Поправка данных за эксцентриситет ВИКИЗ} \label{sect3_12}
Метод ВИКИЗ хорошо зарекомендовал себя в скважинах, пробуренных в Западной
Сибири с применением глинистых буровых растворов. В этих условиях контраст
сопротивлений в среде незначителен. В настоящее время ВИКИЗ все больше
применяется для исследований в скважинах, пробуренных на высокопроводящих
буровых растворах, а также для исследований в разрезах с контрастными по
УЭС горными породами.

При увеличении контраста усиливается влияние геометрической модели среды.
Это приводит к выходу значений кажущегося УЭС за диапазон УЭС в среде.
Классически, при интерпретации данных каротажа применяется модель с зондом
на оси скважины и симметричная цилиндрически слоистая модель среды. А
интерпретация сигналов ВИКИЗ в рамках трехмерных моделей с зондом,
расположенным на стенке скважины, становится практически невозможной из-за
сильного увеличения времени решения прямых задач. Поэтому для интерпретации
данных ВИКИЗ предлагается проводить инверсию в рамках осесимметричной
модели, но перед этим подавлять влияние эксцентриситета на сигналы.

Для подавления влияния эксцентриситета В.С. Игнатовым разработан следующий
алгоритм [13]:
\begin{itemize}
\item На основе трехмерного численного моделирования создаются двухслойные
палетки сигналов при положениях зонда на оси
$\Delta\varphi_{\mbox{\scriptsize о}}$ (синтетический сигнал на оси) и на
стенке $\Delta\varphi_{\mbox{\scriptsize с}}$ (синтетический сигнал на
стенке) скважины от $\rho_{\mbox{\scriptsize п}}$ (УЭС пласта) при
различных $\rho_{\mbox{\scriptsize с}}$ (УЭС скважины) и
$r_{\mbox{\scriptsize с}}$ (радиус скважины).
\item По измеренному на стенке скважины значению
$\Delta\varphi_{\mbox{\scriptsize с}}$ и известным величинам
$\rho_{\mbox{\scriptsize с}}$ и $r_{\mbox{\scriptsize с}}$ по палетке
$\Delta\varphi_{\mbox{\scriptsize c}}(\rho_{\mbox{\scriptsize п}})$
определяется УЭС пласта $\rho_{\mbox{\scriptsize п}}$. Для пластов с
проникновением определенное значение $\rho_{\mbox{\scriptsize п}}$ будет
кажущимся УЭС по модели двухслойной среды «скважина-пласт» (т.е. некоторым
эффективным УЭС среды вне скважины).
\item Затем $\rho_{\mbox{\scriptsize п}}$ по палетке
$\Delta\varphi_{\mbox{\scriptsize c}}(\rho_{\mbox{\scriptsize п}})$ для
известных величин $\rho_{\mbox{\scriptsize с}}$ и 
$r_{\mbox{\scriptsize с}}$ пересчитывается 
в значение $\Delta\varphi_{\mbox{\scriptsize о}}$.
\end{itemize}


Особенностью построения  палетки для данной задачи является малое
количество данных, т.к. время расчета одной трехмерной прямой задачи может
достигать нескольких часов. При этом настройка параметров решателя для
каждой прямой задачи выполняется вручную. Это и обуславливает невозможность
получения сколь угодно плотной расчетной сети для данных палетки.

Расчет данных был разбит на серии по 378 задач. В серии фиксированными
параметрами являются прибор и радиус скважины, причём считается, что зонд
ВИКИЗ расположен на стенке скважины. Зондов в приборе ВИКИЗ "--- 9, а так
же было выбрано 4 радиуса скважины. Итого таких серий было рассчитано 36.
Время исполнения всех задач на одной машине средней производительности
около 70000 часов (это больше 7 лет непрерывной работы). На развёрнутой
нами системе HTCondor расчёты были выполнены за 2 с половиной месяца, при
этом задача считалась только по ночам и в выходные дни [14].

Палетка для процедуры поправки за эксцентриситет строится на основе
решателя прямой задачи, описанном в предыдущей главе. Входными данными
выступают диаметр прибора, радиус скважины, УЭС скважины и УЭС пласта.
Данные (сигнал в секции SI палетки), соответствующие такому набору входных
параметров, "--- сигнал прибора на стенке скважины и соответствующий сигнал
на оси.

Процедура поправки, на основе диаметра прибора, радиуса скважины и УЭС
скважины, определяет все возможные соответствующие УЭС пластов и сигналов,
аппроксимируя по данным в палетке. Затем осуществляется поиск ближайших к
заданному сигналу с прибора на стенке, и интерполяция по соответствующих им
сигналам с оси скважины. Полученное интерполированное значение и будет
результатом поправки за эксцентриситет на основе палетки. Поиск однозначен
в силу природы метода ВИКИЗ, показания прибора монотонны в зависимости от
УЭС пласта двухслойной одномерной модели.


\section{Решение прямой задачи ВИКИЗ}
Для создания решателя прямой задачи ВИКИЗ, ввиду требуемого пространства
моделей и размерности выходного вектора, требуется применить все
инструменты для создания алгоритмов на основе палеток, описанных в
предыдущей главе.

Зачастую, при интерпретации данных ВИКИЗ, значение диэлектрической
проницаемости фиксируют для каждого слоя. Поэтому этот параметр имеет
строго определенное значение. Количество параметров для трехслойной модели
достаточно мало, поэтому расчетная сеть определена так, что она подходит
как для низкоомных (от 0.02 Ом до 0.5 Ом), так и высокоомных (2.0 Ом)
буровых растворов. Наиболее часто радиусы скважины при бурении находятся в
диапазоне от 7 до 20 см. Это учитывается при построении сети. При этом
наиболее часто встречающееся значение "--- это 0.108 м, поэтому сеть
разделена на две пересекающихся (от 0.07 м до 0.108 м и от 0.108 м до 0.2
м). Такие диапазоны параметров для трехслойной модели позволяет оценить
влияние всех параметров УЭС и ширин зон на показания прибора. В таблице 1
описаны основные параметры расчетной сети для трёхслойных моделей ВИКИЗ. По
горизонтали таблица разделена на три части. Каждая из них соответствует
параметрам скважины, ЗП, нетронутой части пласта. По вертикали таблица 1
разбита на четыре подсети. В первой строке описания каждой подсети
обозначено либо значение, либо интервал. На второй "--- мощность интервала.
Внутренние значения в интервале расположены в логарифмическом масштабе.

В таблице 2 заданы параметры четырехслойных моделей ВИКИЗ. В силу
ограничения по памяти у четырехслойных моделей необходимо зафиксировать
несколько параметров. Иначе данных будет так много, что выборка из них
будет происходить существенно дольше, чем это необходимо. Наиболее часто,
исследуемые скважины имеют радиус 0.108 м. Поэтому в этот параметр имеет
фиксированное значение. Дополнительно, палетки построенных на этой сети
ограничены применением только в высокоомных буровых растворах, т.е.
сопротивление скважины имеет фиксированное значение. Таблица 2 имеет
сходную структуру с таблицей 1. Они различаются только наличием параметров
ОЗ.

Расчеты прямых задач ВИКИЗ производились на грид-системе HTCondor,
использующей вычислительные ресурсы ИНГГ СО РАН. Так как время расчета
прямых задач ВИКИЗ составляет порядка 100 мс, то расчетная сеть была
разбита на подсети по 225 тысяч узлов.

\section{Решение прямой задачи БКЗ}
Таблица 3 содержит параметры расчетной сети для трёхслойных моделей БКЗ. В
отличие от таблицы 1 в ней вместо ширины ЗП, задано отношение диаметра ЗП к
диаметру скважины. Это обусловлено программным интерфейсом точного решения
прямой задачи БК. Диапазоны значений сопротивления достаточно велики. Они
охватывают большинство встречающихся в реальных условиях моделей.

\section{Алгоритмы, не требующие реализации палеточных аналогов}
Для того, чтобы библиотека Emfcore имела весь необходимый инструментарий
обработки данных ВИКИЗ и БКЗ, в нее добавлены процедуры расстановки границ
и обратной задачи.

Суть методов расстановки границ заключается в том, чтобы по имеющимся
данным построить кривую, характеризующую возможность наличия границы в
каждой точке. Для имеющихся двух алгоритмов используется разный подход к
формированию этой кривой [15].

Первый это градиентный метод расстановки границ. В своей основе использует
дифференциальные исчисления, взятие производных и выделяет границы в той
области каротажной кривой, которая характеризуется наибольшей скоростью
изменения измеренного параметра. Второй способ расстановки границ "---
дисперсионный метод. В его основе лежат статистические методы определения
границы пласта.

В промышленной обработке данных электрического и электромагнитного каротажа
основным критерием является скорость с достаточной точностью. Поэтому
широкое распространение получили итерационные методы решения обратной
задачи. На основе полученных в ходе зондирования показаний прибора строится
стартовая модель, которая подается на вход прямой задачи для вычисления
теоретических показаний и расчета невязки. Если невязка меньше некоторого
заданного значения, можно считать решение удовлетворительным. Если для
текущей модели теоретические и измеренные показания дают большое
расхождение, модель корректируется и процесс повторяется. Т.е. решение
обратной задачи сводится к многократному решению прямой задачи. Поэтому
использование быстрых прямых задач на основе палеток достаточно ценно.
Решатели обратных задач, которые внедрены в Emfcore, осуществляют
минимизацию функционала невязки методом Нелдера "--- Мида.

Для достоверной оценки фильтрационно-емкостных свойств пласта необходимо
построение как можно более точной геоэлектрической модели. Совместная
инверсия данных по нескольким методам дает возможность построить такую
модель, значительно уменьшив при этом область эквивалентности [16]. В
настоящий момент в библиотеке Emfcore реализован алгоритм решения обратной
задачи, в котором подбор параметров единой модели выполняется сразу по двум
методам.

\section{Количественные характеристики результирующего продукта}

\subsection{Временные характеристики}

Таким образом, как проиллюстрировано в Таблица 4, программная библиотека,
содержащая быстрые алгоритмы для интерпретации, позволяет существенно
сократить время обработки при потоковой обработке данных БКЗ и ВИКИЗ.

Все алгоритмы реализовывались так, чтобы погрешность измерений была меньше
погрешности измерений приборами ЗАО НПП ГА «Луч» для соответствующих
методов.

Решением обратной задачи является множество моделей, сигналы которых
отличаются от искомого в заданных пользователем пределах. В следствии
неединственности решения обратной задачи, в пространстве моделей существуют
области эквивалентных по сигналу моделей, что является следствием как
модельно обусловленной эквивалентностью для метода ВИКИЗ, так и
погрешностями измерений. В результате испытаний было выявлено, что прямая
выборка подходящих моделей из хранилища данных может возвращать более 1000
элементов. При этом,  только небольшая часть из них существенно различаются
друг от друга по модельным параметрам, концентрируясь вокруг
моделей-представителей. В связи с этим, была реализована предобработка
результатов выборки. Для выбора моделей-представителей множество полученных
моделей разбивается на заданное количество кластеров по принципу схожести.
Внутри каждого кластера должны оказаться «похожие» объекты, а объекты
разных кластеров должны быть максимально различаться. Из каждого кластера
выбирается модель-представитель (рис. 5). Полученная совокупность моделей
является результатом решения обратной задачи на основе палеток.

Вычисление точного решения прямой задачи ВИКИЗ производилось алгоритмом
ALVIK [2]. Его средняя скорость решения на обычном современном ПК
(производительность порядка 0.1 Терафлопс) составляет 60 мс для трехслойных
моделей и 90 мс для четырехслойных. Средняя скорость решения с применением
палеток около 1 мс для трех- и четырехслойных моделей. Обратная задача
ВИКИЗ выдает результат спустя 0.1--2 с после запуска, в зависимости от
размера области эквивалентных решений. Для БК эти показатели на два порядка
лучше. Это обусловлено низкой скоростью решения исходной реализацией прямой
задачи.

В связи с полученными временными характеристиками палеточного подхода для
методов электрического каротажа, при совместной интерпретации комплекса
электрометрии в скважинах скорость решения задачи БК будет критическим
фактором. Таким образом, использование палеток при совместной интерпретации
позволит существенно увеличить её производительность работы, а проведение
анализа областей эквивалентности повысит качество заключений.


\section{Варианты реализации: файлы, БД, код} \label{sect3_1}


\section{Описание инструментов для решения задач геоэлектрики} \label{sect3_2}

\section{Примеры решения практических задач: исследовательских, промышленных} \label{sect3_3}


\clearpage